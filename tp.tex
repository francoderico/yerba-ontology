\documentclass[12pt]{report}
\usepackage[utf8]{inputenc}
\usepackage{graphicx}

\usepackage[spanish]{babel}

\usepackage{amsmath,amsthm,amssymb}

\usepackage[margin=1.4in]{geometry}

% Include links
\usepackage{hyperref}

\usepackage{fancyhdr}
\pagestyle{fancy}
\lhead{Introducción a la Inteligencia Artificial} % Left Header
\rhead{Ontología de yerba mate} % Right Header
\cfoot{\thepage} % Center Foot


\title{Ontología de yerba mate}
\author{Santiago Coronel, Franco De Rico, Sebastián Mestre}
\date{Mayo 2023}


\theoremstyle{plain} % other options: definition, remark
\newtheorem*{theorem}{Teorema}
\newtheorem{lemma}{Lema}
\newtheorem*{problem}{Problema}
% Ver tarea Wilson (con proof y todo)




\newcommand{\lmt}[1]{\underset{x\rightarrow {#1}}{\text{lím}}\:}
\newcommand{\lmte}[2]{\underset{{#1}\rightarrow {#2}}{\text{lím}}\:}
\newcommand{\inm}[1]{\in\mathbb {#1}}
\newcommand{\con}[2]{\equiv {#1}\pmod {#2}}
\newcommand{\kmh}{\frac{km}h}
\newcommand{\ms}{\frac ms}
\newcommand{\mss}{\frac m{s^2}}
\newcommand{\vu}[1]{\overset{\rightarrow}{#1}}
\newcommand{\rec}[2]{\overset{\longleftrightarrow}{{#1}{#2}}}
\newcommand{\srec}[2]{\overset{\longrightarrow}{{#1}{#2}}}
\newcommand{\seg}[2]{\overline{{#1}{#2}}}
\newcommand{\dv}[2]{\frac{\mathrm{d}{#1}}{\mathrm{d}{#2}}}
\newcommand{\pd}[2]{\frac{\partial{#1}}{\partial{#2}}}

%\begin{figure}[h]
%\includegraphics[width=\textwidth]{a}
%\end{figure}

\begin{document}

\maketitle
\eject

\section*{Descripción}

La ontología presentada pretende representar conocimiento en el dominio de la yerba mate. Consideramos que es un campo suficientemente amplio, ya que deben describirse a su vez las propiedades organolépticas de los distintos tipos de yerba, las distintas formas de producción de la misma, y las diversas marcas que la comercializan.

Por esto se decidió utilizar las siguientes clases:

- Yerba: cada individuo de esta clase es un producto específico encontrado en el mercado. Pueden subclasificarse según tipos de producción (por ej., barbacuá), o por propiedades del producto en sí (por ej., orgánica o saborizada).

- Fruta: dado que es muy común encontrar yerbas saborizadas con distintos tipos de fruta, se decidió crear una clase específica para estas.

- Yuyo: al igual que la clase anterior, es usual encontrar productos con agregado de hierbas, por lo que surge naturalmente el uso de esta clase para representarlas.

- Sabor: fue definida para representar los sabores más comunes encontrados en el mercado. Inicialmente es instanciada con Amargo, Fuerte, Herbal y Suave. De acuerdo a distintas necesidades podrían agregarse más instancias para representar distintos sabores.

- TipoCosecha y TipoElaboración: se definieron estas clases para representar rasgos propios del proceso productivo, tal como sus nombres lo indican.

- Además fueron definidas las clases Marca, Precio, y Región.

\bigskip
Fue considerada la opción de importar una ontología para las frutas o las hierbas, pero no se encontró ninguna que cumpliera las expectativas.

Algunas preguntas que interesa que la ontología sea capaz de resolver son:

¿Qué formas de cosecha disminuyen el precio final del producto? 

\textit{cosechaImplicaPrecio value Bajo}

¿Qué yerbas tienen precio alto?

\textit{(yerbaTienePrecio value Alto) or (yerbaTipoCosecha some (cosechaImplicaPrecio value Alto))}

¿Cuáles son las yerbas con palo?

\textit{Con\_palo}

¿Cuáles son sin palo?

\textit{Sin\_palo}

¿Qué marcas producen variedades orgánicas?

\textit{marcaProduceYerba some Orgánica}

\bigskip

Notar que en algunos casos la query es compleja, mientras que en otros el trabajo ya fue hecho al definir la clase, permitiendo consultas muy simples.

\section*{Gráficos}

Se presentan a continuación el diagrama de las clases de la ontología (en el que aparecen además las relaciones que fueron consideradas más importantes), y un diagrama con algunas instancias de ejemplo, y sus relaciones. En este último, las relaciones con línea punteada son inferidas.

%\begin{figure}[h]
\includegraphics[width=\textwidth]{grafo_yerba.jpg}
%\end{figure}

%\begin{figure}[h]
\includegraphics[width=\textwidth]{graphviz.png}
%\end{figure}

\section*{Conclusión}

En líneas generales se considera que la ontología desarrollada tiene un tamaño lo suficientemente amplio como permitir ciertos niveles de expresividad, pudiendo representarse distintos puntos en el sabor, componentes y producción de la yerba mate.

Durante el desarrollo fue ventajoso contar con bibliografía relativamente extensa en la web. Esto permitió un trabajo fluido y sin mayores contratiempos; sin embargo, se listan a continuación algunos puntos de interés a futuro y dificultades que fueron encontradas:

- Existe cierta complejidad en representar la relación yerba - marca. Una marca puede producir distintos tipos de yerba; un tipo de yerba puede ser producido por varias marcas o ser exclusivo de una única marca. La ontología debería poder adaptarse fácilmente a que una yerba previamente exclusiva pase a ser producida por varias marcas.

- En una primera etapa, se encontraron incovenientes en la definición de ciertos conceptos (por ejemplo, los tipos de cosecha), debiendo decidir si se incluían como subclases o como individuos.

- Como trabajo futuro, sería interesante poder inferir información sobre el sabor de la yerba a partir del sabor de sus componentes y agregados. Las reglas deberían contemplar cierto nivel de complejidad; se precisaría conocimiento experto. Sería útil contar con mas bases de conocimiento para enriquecer la información sobre los agregados (frutas, hierbas, etc.).

\section*{Referencias}

\href{https://www.taragui.com/aprender/historia/produccion-de-la-yerba-mate}{Producción de la yerba mate}

\href{https://www.mateologist.com/tipos-de-yerba-mate/}{Tipos de yerba mate}

\href{https://yerbamateargentina.org.ar/es/noticias/consejos-materos/79020-tipos-de-yerba-mate-y-como-elegirlas.html}{Tipos de yerba mate y cómo elegirlas}

\href{https://inym.org.ar/descargar.html?archivo=enR1K1ZxRDB1czZKeW80L01ZYVpJZz09}{Yerba mate: Manual de producción}

\end{document}
